%kkkkkkkkkkkkkkkkkkkkkkkkkkkkkkkkkkkkkkkkkkkkkkkkkkkkkkkkkkkkkkkkkkkkkkkkkkkkkkk
\documentclass[12pt]{article}
\usepackage{graphicx,amssymb}

\begin{document}
\section{Questions of investigation}
The North American river otter (\textit{Lontra canadensis}) is endemic to all of 
Minnesota; however, populations declined precipitously in the 19th and 20th 
centuries due to trapping and habitat degradation. In recent years their 
populations have significantly rebounded, especially in southeastern Minnesota 
where hunters would like to be able to trap otters again. In order for that to 
be possible governmental organizations must be able to monitor the population 
health. In addition, otter presence can indicate high quality aquatic habitat. 
Due to these reasons, there is interest in monitoring changes in otter abundance 
over time. Unfortunately, otters are not easy to observe. Thus, this study 
investigated the potential of using wintertime aerial surveys of otter tracks on 
rivers to make inferences about the abundance of otters and how it changes over 
time. We used the concepts from occupancy theory to develop models to estimate 
abundance.

\subsection{Data collection}
During the winters of 2003 and 2004 people flew over the Whitewater, Zumbrow, 
and Mississippi Rivers. When they saw what looked like an otter track, they 
pushed a button recording the GPS coordinates. They pushed the button every five 
seconds until the track ended. Flights occurred one, two, or three days after a 
snow event. In some cases, multiple observers flew over the same river on the 
same day. Surveys were conducted during multiple snow events throughout both 
winters. The rivers were then divided into 400 and 800 meter plots and the 
presence (1) or absence (0) of a waypoint in a plot on a given flight was 
determined. The Mississippi River was also divided into 1600 meter plots.

\subsection{Issues related to the data}
One major issue with respect to the data is that if a track occurred near the 
boundary between two plots, a person on one flight may have recorded the track 
in one site whereas someone on a later flight during that same snow event may 
have seen the same track but recorded it in the neighboring plot. So while the 
data indicate that the observers on the two flights disagreed in two plots, they 
saw the same track. Increasing the plot size should decrease the magnitude of 
this effect.

In addition, there is concern that the probability of an otter laying down a 
track in one site is not independent of the probability that an otter lays down 
a track in neighboring sites, violating a principle assumption of occupancy 
theory. That is, a single track can extend across multiple plots. This leads to 
dependence among the plots. It was important to quantify this dependence and 
find an optimum way to handle it.

\subsection{Descriptive statistics}

In 2003, three different observers flew over the rivers. Assuming that no tracks 
are laid down between individuals� flights on the same day, all three people 
should have seen the same tracks. While there is some error in classifying which 
site the observer saw the actual track in, each observer should at least be 
seeing the same, or nearly the same, number of sites with tracks. However, we 
did not find that to be the case. If we look only at days that all three people 
observed, there appear to be major differences among the proportion of sites in 
which the scientists detected a track (Figure x). The difference in estimated na
\"ive occupancy among the observers indicates that the scientists did not always 
correctly identify the tracks. 

Correlation among the flight sequences is low. For example, average correlation 
between flights on the Mississippi in 2003 is 0.176 for 400 meter plots. Some of 
this is due to the difficulties in determining to which site each track belongs. 
However, we have also found quite a bit of variation in the data not necessarily 
due to these classification issues. This is difficult to quantify, as an 
algorithm for finding classification errors would be complicated. Correlation 
between flights is a bit higher for flights on the same day or flights during 
the same snow event, but still not as high as one might hope. The proportion of 
sites with tracks detected generally increases with days since snow, likely 
because the otters have a longer time to lay down a track as time after the snow 
event passes.

\subsection{False positive rates}
The lack of consistency among the flights is strong evidence that the data 
contain false positives where scientists think they see a track when it does not 
exist. Not accounting for such a process can lead to major biases; as the number 
of flights increases, the occupancy rate estimate will approach one (Royle and 
Link 2006). We introduced a false detection process and used the observer as a 
covariate (by means of indicator variables) for both the probability of 
correctly identifying a track that exists ($p$) and the probability of 
identifying a track that does not exist ($E$). Including false positives 
introduces a duality whereby $\begin{array}{ccc}(p = x, & E = y, & \psi = z)\end
{array}$ is as good a solution as $\begin{array}{ccc}(p = y, & E = x, & \psi = 
1-z)\end{array}$. To address this problem we assumed $p\gg0.5$ and $E\ll0.5$, an 
approach recommended by Royle and Link (2006).

Figure x caption:
The box plots show how the proportion of sites in which an observer saw tracks 
(�occupancy�) differs among the three observers. Only those dates in which all 
three observers flew are included.

\section{Spatial correlation of the data}
We found that there is significant correlation between the probability of 
observing a track in neighboring sites, but this violates of a central 
assumption of occupancy theory. This issue needed to be addressed. One approach 
was to make the prior distribution of $\theta$, the probability that an otter 
lays down a track, spatially correlated using a conditional autoregressive
(CAR) model.



\end{document}