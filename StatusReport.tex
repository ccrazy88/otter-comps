% StatusReport.tex
% Chrisna Aing and Sarah Halls

\documentclass[12pt]{article}
\usepackage{amsmath}
\usepackage{multirow}
\usepackage[hmargin=3cm, vmargin=3cm]{geometry}
\begin{document}

    \noindent{\textbf{Status Report for Otter Estimation Project \hfill Feb. 18,
    2010}
    
    \noindent{From: Chrisna Aing, Sarah Halls, Kiva Oken, Carleton College
    Senior Comps group} \\
    
    Recall that our "standard model," contains the following parameters:
    \(\psi\) (occupancy rate), \(\theta\) (probability of laying down a track,
    given occupancy), \(p\) (the detection probability for each observer; so \(1
    - p\) is the false negative rate), and \(E\) (the false positive rate for
    each observer).  We have also worked with a CAR model using the WinBUGS CAR
    installation.

    Since both our standard model and our CAR model have been working relatively
    well on our simulated data, we moved onto running them on the actual data.

    The standard model ran without convergence issues on the 2003 data for all
    rivers, but it did not converge on any of the 2004 data.  We simplified the
    model by eliminating E (assuming no false positives) in an attempt to
    improve convergence, which worked for the 402m Whitewater River and 402m 
    Zumbro River data.  Although none of the Mississippi River's 2004 data 
    converged, we are still including estimates for the 402m Mississippi River 
    data.

    The CAR model did not converge for any of the data.  Although the standard
    model and the CAR model performed similarly in testing when both models
    converged, the standard model converged much more often than the CAR model
    did.  We think this is because the CAR model attempts to estimate too many
    parameters.  There is a different estimate of \(\theta\) for every site,
    which adds a considerable amount of complexity to the model.  We also tried
    to simplify this model by eliminating E because this was somewhat successful
    with the standard model.  This improved convergence with the 402m data, but
    gave us \(\psi\) estimates greater than 0.9 for all rivers, which does not
    seem reasonable.  Therefore, we are not going to include any estimates of
    parameters from the CAR model in this report.

    We decided not to run the models on the alternating plots.  In our earlier
    analysis, we discovered that the main difference between alternating plots
    and all plots was that the alternating plots showed more independence in the
    data.  When testing the standard model on spatially dependent simulated
    data, we found that the estimates were fairly accurate.  The CAR model
    estimates were also fairly accurate on this simulated data, when it
    converged.  Therefore, we decided to use the data that includes all plots
    because we feel that having the additional information from all plots is
    more important to the model than independence is.

    All of our estimates are below and are accompanied with some thoughts about
    their convergence.  We are wondering if these estimates seem reasonable,
    based upon what you know about the three rivers. \\

    % Mississippi River!
    \begin{tabular}{|l|l|l|l|}
        \hline
        \multicolumn{4}{|l|}{\textbf{Mississippi River, 2003}} \\
        \hline
            & 402m & 804m & 1608m \\
        \hline
        \multirow{2}{*}{E[Danny]}
            & 0.01954 & 0.04808 & 0.06885 \\
            & (0.006515, 0.03663) & (0.01874, 0.08187) & (0.01778, 0.1383) \\
        \hline
        \multirow{2}{*}{E[John]}
            & 0.05216 & 0.07771 & 0.13650 \\
            & (0.027710, 0.07745) & (0.03881, 0.12100) & (0.03449, 0.2330) \\
        \hline
        \multirow{2}{*}{E[Tom]}
            & 0.02668 & 0.06140 & 0.10230 \\
            & (0.011990, 0.04622) & (0.02789, 0.10390) & (0.04287, 0.1796) \\
        \hline
        \multirow{2}{*}{p[Danny]}
            & 0.65240 & 0.67340 & 0.74960 \\
            & (0.548300, 0.75990) & (0.56760, 0.78290) & (0.60510, 0.8910) \\
        \hline
        \multirow{2}{*}{p[John]}
            & 0.67770 & 0.77580 & 0.82540 \\
            & (0.583500, 0.77280) & (0.66060, 0.88690) & (0.71390, 0.9326) \\
        \hline
        \multirow{2}{*}{p[Tom]}
            & 0.53670 & 0.54220 & 0.65330 \\
            & (0.501200, 0.61300) & (0.50160, 0.62850) & (0.51530, 0.8104) \\
        \hline
        \multirow{2}{*}{\(\psi\)}
            & 0.59380 & 0.69370 & 0.75820 \\
            & (0.439000, 0.75990) & (0.51800, 0.87380) & (0.54110, 0.9300) \\
        \hline
        \multirow{2}{*}{\(\theta\)}
            & 0.15280 & 0.20730 & 0.27690 \\
            & (0.111900, 0.19780) & (0.14580, 0.27390) & (0.19030, 0.3716) \\
        \hline
    \end{tabular}
    \\
    
    This data converged well, for the most part.  At the 1608m plot size, E,
    \(\psi\), and \(\theta\) did not converge well. \\

    \begin{tabular}{|l|l|}
        \hline
        \multicolumn{2}{|l|}{\textbf{Mississippi River, 2004}} \\
        \hline
            & 402m \\
        \hline
        \multirow{2}{*}{p[Danny]}
            & 0.3129 \\
            & (0.2174, 0.8107) \\
        \hline
        \multirow{2}{*}{\(\psi\)}
            & 0.7613 \\
            & (0.6609, 0.8833) \\
        \hline
        \multirow{2}{*}{\(\theta\)}
            & 0.8511 \\
            & (0.2426, 0.9973) \\
        \hline
    \end{tabular}
    \\

    This data did not converge.  We do not think that these estimates are
    accurate, but they are included for the sake of completeness. \\
    
    % Whitewater River!
    \begin{tabular}{|l|l|l|}
        \hline
        \multicolumn{3}{|l|}{\textbf{Whitewater River, 2003}} \\
        \hline
            & 402m & 804m \\
        \hline
        \multirow{2}{*}{E[Danny]}
            & 0.05608 & 0.0555 \\
            & (0.02474, 0.09554) & (0.002859, 0.1390) \\
        \hline
        \multirow{2}{*}{E[John]}
            & 0.16570 & 0.1931 \\
            & (0.10130, 0.23510) & (0.065910, 0.3384) \\
        \hline
        \multirow{2}{*}{E[Tom]}
            & 0.15350 & 0.2241 \\
            & (0.09758, 0.21660) & (0.110000, 0.3623) \\
        \hline
        \multirow{2}{*}{p[Danny]}
            & 0.66990 & 0.6985 \\
            & (0.56080, 0.79040) & (0.573100, 0.8366) \\
        \hline
        \multirow{2}{*}{p[John]}
            & 0.92040 & 0.9877 \\
            & (0.84270, 0.97590) & (0.945100, 1.0000) \\
        \hline
        \multirow{2}{*}{p[Tom]}
            & 0.75910 & 0.8575 \\
            & (0.65060, 0.85800) & (0.745400, 0.9600) \\
        \hline
        \multirow{2}{*}{\(\psi\)}
            & 0.87590 & 0.9126 \\
            & (0.72910, 0.98880) & (0.746900, 0.9955) \\
        \hline
        \multirow{2}{*}{\(\theta\)}
            & 0.24100 & 0.3779 \\
            & (0.17890, 0.30900) & (0.286500, 0.4721) \\
        \hline
    \end{tabular}
    \\

    While the 402m data converged well, E, \(\psi\), and \(\theta\) did not
    converge well for the 804m data. \\

    \begin{tabular}{|l|l|l|l|}
        \hline
        \multicolumn{2}{|l|}{\textbf{Whitewater River, 2004}} \\
        \hline
            & 402m \\
        \hline
        \multirow{2}{*}{p[Danny]}
            & 0.8237 \\
            & (0.5323, 0.9804) \\
        \hline
        \multirow{2}{*}{\(\psi\)}
            & 0.6066 \\
            & (0.4246, 0.8382) \\
        \hline
        \multirow{2}{*}{\(\theta\)}
            & 0.2068 \\
            & (0.1211, 0.3550) \\
        \hline
    \end{tabular}
    \\

    This data converged well.  However, we feel that the absence of E from the
    model makes the estimates much less dependable. \\

    % Zumbro River!
    \begin{tabular}{|l|l|l|}
        \hline
        \multicolumn{3}{|l|}{\textbf{Zumbro River, 2003}} \\
        \hline
            & 402m & 804m \\
        \hline
        \multirow{2}{*}{E[Danny]}
            & 0.02922 & 0.007066 \\
            & (0.006786, 0.05994) & (0.0000, 0.03403) \\
        \hline
        \multirow{2}{*}{E[John]}
            & 0.03388 & 0.008312 \\
            & (0.008584, 0.06880) & (0.0000, 0.04188) \\
        \hline
        \multirow{2}{*}{E[Tom]}
            & 0.02228 & 0.012960 \\
            & (0.002430, 0.05385) & (0.0000, 0.05642) \\
        \hline
        \multirow{2}{*}{p[Danny]}
            & 0.81890 & 0.837100 \\
            & (0.749500, 0.88540) & (0.7615, 0.90700) \\
        \hline
        \multirow{2}{*}{p[John]}
            & 0.81770 & 0.852500 \\
            & (0.740100, 0.88670) & (0.7597, 0.92230) \\
        \hline
        \multirow{2}{*}{p[Tom]}
            & 0.73120 & 0.805400 \\
            & (0.625400, 0.82440) & (0.6885, 0.89780) \\
        \hline
        \multirow{2}{*}{\(\psi\)}
            & 0.66550 & 0.651600 \\
            & (0.549100, 0.78150) & (0.5098, 0.78210) \\
        \hline
        \multirow{2}{*}{\(\theta\)}
            & 0.29950 & 0.395800 \\
            & (0.236400, 0.36550) & (0.3100, 0.48520) \\
        \hline
    \end{tabular}
    \\

    Again, the 402m data converged well.  With respect to the 804m data, while
    \(\psi\) and \(\theta\) converged really well, none of the other parameters
    did. \\

    \begin{tabular}{|l|l|l|l|}
        \hline
        \multicolumn{2}{|l|}{\textbf{Zumbro River, 2004}} \\
        \hline
            & 402m \\
        \hline
        \multirow{2}{*}{p[Danny]}
            & 0.9166 \\
            & (0.7739, 0.9918) \\
        \hline
        \multirow{2}{*}{\(\psi\)}
            & 0.4773 \\
            & (0.3586, 0.6114) \\
        \hline
        \multirow{2}{*}{\(\theta\)}
            & 0.4232 \\
            & (0.3073, 0.5520) \\
        \hline
    \end{tabular}
    \\

    This data converged well.  However, as is the case with all of the 2004
    data, we feel that the absence of E from the model makes the estimates much
    less dependable. \\

\end{document}