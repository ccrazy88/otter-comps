\documentclass{article}
\usepackage{amsmath}
\usepackage{fullpage}
\usepackage[hmargin=3cm, vmargin=3cm]{geometry}
\begin{document}

\section{Overview of Occupancy Modeling}

	\subsection{What is Occupancy Rate?}

		Occupancy is "The probability that a randomly selected site or sampling 
unit in an area of interest is occupied by a species" [MacKenzie2006] This is 
not the same as the abundance, or population size, of a species.  Instead of 
measuring the number of animals present in a species, occupancy measures the 
proportion of land that is used as the habitat of at least one member of the 
species. 
	is either the probability or the proportion 25-  note importance of the 
distinction although they can be considered the same thing in most cases 
		
		*What is it used for?

	\subsection{Different Ways of Sampling Data}
		Most common way of sampling:  visit sites and spend time within each one 
looking for either individuals of the species or for evidence the species is 
present- "presence-absence survey"(26).  can use visual observation of animals,
capture animals, ovserve tracks, detect vocalizations, camera traps, sound 
recording.  leads to naive occupancy rate.  PROBLEM: false negatives.
		
		*Measuring rate of occupancy vs. population \\
		*Looking at tracks vs. listening to stuff or acutally looking for the 
			animal

	\subsection{Assumptions}
		*Independence of detection between sites \\
		*No false positives \\
		*Presence and detection probabilities constant across sites and surveys \\
		*Closed population

	\subsection{Types of Occupancy Models}
		We only considered single-season, single-species models because this is 
the data we had, but any of these models can be extended to take into account
multiple seasons or multiple species.  
		Naive Occupancy pg 11- this is when you just divide the number of sites
			that are indicated as occupied by the total number of sites.  it is
			generally accepted that this is too low because not all animals are 
			detected.
		Hierarchical Models:
		Maximum Likelihood \\  Give the simple model, then say it can be 
			modified in many different ways.  give some examples possibly 
			without going into detail.
		Any other models? \\
		Bayesian?  is that the name of what we used?  should save most of this 
			for theoretical section?

\section{Actual data}
Stuff by Kiva.

	\subsection{Markov Chains and Testing the Independence Assumption}

		First we considered using Markov Chains to come up with a long term 
probability of detection, which we called \(r\).  We let \(p\) be the number of 
\(0\)'s followed by a \(0\) as a fraction of all successive pairs that start 
with a \(0\).  We let \(q\) be the number of \(1\)'s followed by a \(1\) as a 
fraction of all successive pairs that start with a \(1\).  These probabilities 
can be represented by the matrix \(M = \begin{bmatrix} p & 1-p \\ 1-q & q 
\end{bmatrix}\). The long term probability of detection can be found from this 
by finding \(\lim_{n \to \inf} M^n\), which returns \( \begin{bmatrix} 1-r & r 
\\ 1-r & r\end{bmatrix}\), where \(r\) is the probability of a \(1\), which is 
the long term probability of detection.  

	We calculated \(p\), \(q\), and \(r\) for every flight in our data, using 
\(n = 500\).  We then realized that \(r\) does not answer the questions we are 
asking.  It does not take into account whether or not a site is occupied, or 
whether or not a track was laid down on that site, or any of the ideas behind    
the hierarchical model.  These are the parameters that we eventually want to 
find out something about, and the value of \(r\) cannot help us get there.  The 
ideas behind using Markov Chains did end up being useful, however.  
	
	Although we rejected the idea of using \(r\) as a way to analyze the data, 
we realized that \(p\) and \(q\) could still be used to test the viability of 
the independence assumption.  If there is independece between sites, the 
difference of the proportions \(p\) and \(1-q\) should be approximately \(0\). 
This would mean that the probability that a \(0\) follows a \(0\) is about the 
same as the probability that a \(1\) follows a \(0\), meaning that it does not 
depend on what the observation is at the previous site.  Looking at boxplots of 
\(p - (1-q)\), we see that data from flights that record alternating sites may 
be more independent than data from flights that record all sites, which suggests
a possible benefit to using the alternating sampling method.  The boxplots for 
the Whitewater and Zumbro Rivers looked similar to the Mississippi:  \\
*INSERT BOXPLOT FROM STATUS REPORT HERE  \\
We found no obvious trend indicating one site size to be more independent than 
the others.  

	While this is an interesting observation, we would like to be able to 
quantify it better by finding the distribution of our statistic.  This would 
allow us to make statements about how independent the data is, and if the 
difference in independence between different types of data is significant.   
This statistic seems hard to get, but we considered a modification.  Let \(S\) 
be the number of successive \(0\),\(1\)'s minus the number of successive \(1\),
\(1\)'s per flight.  Under the null hypothesis that each site is an independent 
Bernoulli trial with probability \(p\) we were able to compute the mean and 
standard deviation of \(S\), which we used to calculate a Z-score.
	In order to compute these values, we needed to find the expected value and
variance of each term in \(S\).  We let \(A = \text{\# of 0-1's}\), 
\(B = \text{\# of 1-1's}\) and \(p = P(``1")\text{ in the sequence}\).

	The mean ...
 	\begin{align*}
        E[A] 
            &= \sum_{i=1}^{n-1} P(X_i = 0, X_{i+1} = 1) \\
            &= \sum_{i=1}^{n-1} (1-p)p \\
            &= (n-1)(1-p)p
    \end{align*}
    \begin{align*}
        E[B]
            &= \sum_{i=1}^{n-1} P(X_i = 1, X_{i+1} = 1) \\
            &= \sum_{i=1}^{n-1} p*p \\
            &= (n-1)p^2
    \end{align*}
	\begin{align*}
        E[\hat{s}] 
            &= E[A] - E[B] \\
            &= (n-1)\bigg((1-p)p - p^2\bigg)
    \end{align*}

	The standard deviation...
	\begin{align*}
        Var[A] 
            &= \sum_{i=1}^{n-1} Var[A] + 2\sum_{i=1}^{n-2} Cov[I_i = (0, 1),
               I_{i+1} = (1, 1)] \\
            &= (n-1)P((0, 1))(1-P((0, 1))) + 2\sum_{i=1}^{n-2}\bigg(E[I_i=(0, 
               1), I_{i+1} = (0, 1)] - E[I_i = (0, 1)]E[I_{i+1} = (0, 1)]\bigg)
               \\
            &= (n-1)(1-p)p(1-(1-p)p) + 2\sum_{i=1}^{n-2}\bigg(0 - (1-p)p(1-p)p
               \bigg) \\
            &= (1-p)p(-1+n+5p-3np-5p^2+3np^2)
    \end{align*}
    \begin{align*}
        Var[B] 
            &= \sum_{i=1}^{n-1} Var[B] + 2\sum_{i=1}^{n-2} Cov[I_i = (1, 1),
               I_{i+1} = (1, 1)] \\
            &= (n-1)P((1, 1))(1-P((1, 1))) + 2\sum_{i=1}^{n-2}\bigg(E[I_i=(1,   
               1), I_{i+1} = (1, 1)] - E[I_i = (1, 1)]E[I_{i+1} = (1, 1)]\bigg) 
               \\
            &= (n-1)p^2(1-p^2) + 2(n-2)(p^3-p^4) \\
            &= (1-p)p^2(-1+n-5p+3np)
    \end{align*}
	\begin{align*}
		2Cov[A, B]  
			&= 2 \sum_{i=1}^{n-1} \sum_{j=1}^{n-1} Cov[A, B]\\
			&= 2 \bigg(\sum_{i=1}^{n-2} Cov[AB] + \sum_{i=1}^{n-2} Cov[BA] +
				\\ \sum_{i=1}^{n-1} Cov[X_i = 0 \text{ or } 1, X_{i+1} = 1]\bigg) \\
			&= 2 \bigg(\sum_{i=1}^{n-2} E[AB] - E[A]E[B] + 
				\sum_{i=1}^{n-2} E[BA] - E[B]E[A] + \\
				\sum_{i=1}^{n-1} E[X_i = 0 \text{ or } 1, X_{i+1} = 1] - E[B]E[A] \\
			&= 2(n-2)(p-1)^2p^2 - 2(n-2)p^3(1-p) - 2(n-1)p^3(1-p)			
	\end{align*}	
    \begin{align*}
        Var[\hat{s}] 
            &= Var[A-B] \\
            &= Var[A] + Var[B] - 2Cov[A, B] \\
            &= (1-p)p(-1+n+8p-4np-20p^2+12np^2)
    \end{align*}

	Simulations of \(S\) show that the distribution of \(S\) is roughly normal.  
So using our calculations for the mean and the standard deviation, we computed
a Z-score for each flight, where we use a plug-in estimate for \(p\) to be the 
proportion of \(1\)'s per flight.  We said that Z-scores with an absolute value 
greater than \(2\) suggest that the assumption of independence between sites is 
violated for that flight.  We found that, in general, alternating sites are more
"independent" than all sites as measured by this statistic.  However, alternating
sites do not guarentee independence.  For instance, six of eleven flights along 
the Zumbro River during 2003 still have significant Z-scores with alternating
sites.  There were no noticable trends for this statistic with respect to 
observer, days since snow, snow event, or plot size.

*FIGURES OR TABLES??? WHICH ONES?

	We did find interesting trends with regard to sampling method.  We think 
that the lack of independence results from strings of ones where an otter left a
track across multiple sites.  When alternating sites are used, the strings are 
cut in half and the occurrence of \(1\),\(1\)'s is lower while the occurence of 
\(0\),\(1\)'s stays about the same.  It is noteworthy that virtually all the 
violations of independence are in the direction of more \(1\),\(1\)'s than would
be expected.  This reinforces our thoughts about successive \(1\)'s representing
the same otter.  We will discuss later whether this violation of independence 
actually matters in our model, and how this might affect the estimation of 
actual occupancy rates.

\section{Theoretical Stuff} 

	\subsection{Bayesian Statistics}
	   	Bayes' Theorem \\
    	Useful for hierarchical modeling: look at section 3.3 in book. \\
    	Can use Markov Chain Monte Carlo (MCMC) to calculate posterior 
			distribution- bimodal issues.
	
	\subsection{BRugs/R} - Chrisna

\end{document}