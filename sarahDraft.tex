\documentclass{article}
\usepackage{amsmath}
\begin{document}

\section{Overview of Occupancy Modeling}
	\subsection{What is Occupancy Rate?}
		"The probability that a randomly selected site or sampling unit in an area 
			of interest is occupied by a species" [MacKenzie2006] (pg. 11)
	\subsection{Different Ways of Sampling Data}
		Measuring rate of occupancy vs. population \\
		Looking at tracks vs. listening to stuff or acutally looking for the animal
	\subsection{Assumptions}
		Independence of detection between sites \\
		No false positives \\
		Presence and detection probabilities constant across sites and surveys \\
		Closed population
	\subsection{Types of Occupancy Models}
		Maximum Likelihood \\
		any other models? \\
		Hierarchical Bayesian?  is that the name of what we used?

\section{Actual data}
Stuff by Kiva.
	\subsection{Markov Chains and Testing the Independence Assumption}
		First we considered using Markov Chains to come up with a long term 
probability of detection, which we called \(r\).  We let \(p\) be the number of 
\(0\)'s followed by a \(1\) as a fraction of all successive pairs that start with 
a \(0\).  We let \(q\) be the number of \(1\)'s followed by a \(1\) as a fraction 
of all successive pairs that start with a \(1\).  

		We wanted to find out if the assumption of independence across sites held
true for our data from the DNR.  We also wanted to see what sampling techniques 
led to the greatest independence between plots.  The different techniques that we
compared were using different plot sizes, and using data that includes all plots
or data that uses only every other plot. \\
		(We tried to see if the assumption of independence is viable and whether or
			not there were differences between data with different plot sizes and 
			differences between data containing plots and data containing every 
			other plot.) \\
		If the observations at different sites are independent, then the 
probability that there is a \(1\) at site \(i\)  is independent of site \(i-1\).  
We let \(p\) be the number of \(0\)'s followed by a \(1\) as a fraction of all 
successive pairs that start with a \(0\).  We let \(1-q\) be the number of \(1\)'s
followed by a \(1\) as a fraction of all successive pairs that start with a \(1\).
The difference of these two proportions \((=p-(1-q)\) is a measure of independence,
where \(p-(1-q)\) being approximately \(0\) under the null hypothesis of 
independence.  
	\begin{enumerate}
    	\item Using Markov Chains to find r. \\
          * r = potential long-term probability of detection, based upon the 
			short-term conditional probabilities p and q acquired from our data 
			(infinite limit as n approaches infinity of Pn) \\
          * p = probability that a 0 follows a 0, q = probability that a 1 follows
			 a 1  \\
          * If 0->0 and 1->0 (the values of a column, basically) equaled each 
			other, plots would have been independent?  \\
          * Ultimately not useful to us because it does not take into account 
			whether or not a site is occupied, so it doesn't apply to the 
			hierarchical model that we are using. \\
   		\item Using s, where s = (\# of 0-1's) - (\# of 1-1's) for a given flight.\\
          * Decided to use the z-score of s as a measure of independence.  \\
          * Therefore, needed the expected value and the standard deviation of s. \\
          * So, we needed the expected value of each term and the variance, which 
			required finding the variance of each term and the covariance between 
			the two terms. \\
          * Expected value is straightforward and we have all of the equations. \\
          * Variance is fun and we also have all of the relevant equations. \\
          * Calculated z-score for every flight and box-plotted them for each year
			and plot size and pitted data with every other plot against data with
			all plots. \\
          * Nothing important with respect to plot size. Alternating plots show 
			more independence than all plots. Look at report for more. \\
          * CAR model started working once we took s values of simulated data into
			account by increasing dependence.  \\
	\end{enumerate}

\section{Theoretical Stuff} - CHRISNA??
	\subsection{Bayesian Statistics}
	   	Bayes' Theorem \\
    	Useful for hierarchical modeling: look at section 3.3 in book. \\
    	Can use Markov Chain Monte Carlo (MCMC) to calculate posterior 
			distribution- bimodal issues.
	\subsection{BRugs/R}
    	All of our statistical analysis was performed in R.  \\
    	All of our modeling was performed using BRugs, which allowed us to use 
			WinBUGS commands and stuff in R.


\end{document}