\documentclass{article}
\usepackage{amsmath}
\usepackage{fullpage}
\begin{document}

\section{Overview of Occupancy Modeling}
	\subsection{What is Occupancy Rate?}
		"The probability that a randomly selected site or sampling unit in an area 
			of interest is occupied by a species" [MacKenzie2006] (pg. 11)
	\subsection{Different Ways of Sampling Data}
		Measuring rate of occupancy vs. population \\
		Looking at tracks vs. listening to stuff or acutally looking for the animal
	\subsection{Assumptions}
		Independence of detection between sites \\
		No false positives \\
		Presence and detection probabilities constant across sites and surveys \\
		Closed population
	\subsection{Types of Occupancy Models}
		Maximum Likelihood \\
		any other models? \\
		Hierarchical Bayesian?  is that the name of what we used?

\section{Actual data}
Stuff by Kiva.
	\subsection{Markov Chains and Testing the Independence Assumption}

		First we considered using Markov Chains to come up with a long term 
probability of detection, which we called \(r\).  We let \(p\) be the number of 
\(0\)'s followed by a \(0\) as a fraction of all successive pairs that start with 
a \(0\).  We let \(q\) be the number of \(1\)'s followed by a \(1\) as a fraction 
of all successive pairs that start with a \(1\).  These probabilities can be
represented by the matrix \(M = \begin{bmatrix} p & 1-p \\ 1-q & q 
\end{bmatrix}\). The long term probability of detection can be found from this by finding 
\(\lim_{n \to \inf} M^n\), which returns \( \begin{bmatrix} 1-r & r \\ 1-r & r 
\end{bmatrix}\), where \(r\) is the probability of a \(1\), which is the long 
term probability of detection.  

	We calculated \(p\), \(q\), and \(r\) for every flight in our data, using 
\(n = 500\).  We then realized that \(r\) does not answer the questions we are 
asking.  It does not take into account whether or not a site is occupied, or 
whether or not a track was laid down on that site, or any of the ideas behind the
hierarchical model.  Thesse are the parameters that we eventually want to find 
out something about, and the value of \(r\) cannot help us get there.  The ideas 
behind using Markov Chains did end up being useful, however.  
	
	Although we rejected the idea of using \(r\) as a way to analyze the data, we
realized that \(p\) and \(q\) could still be used to test the viability of the 
independence assumption.  If there is independece between sites, the difference 
of the proportions \(p\) and \(1-q\) should be approximately \(0\).  This would 
mean that the probability that a \(0\) follows a \(0\) is about the same as the 
probability that a \(1\) follows a \(0\), meaning that it does not depend on what
the observation is at the previous site.  Looking at boxplots of \(p - (1-q)\), 
we see that data from flights that record alternating sites may be more 
independent than data from flights that record all sites, which suggests a 
possible benefit to using the alternating sampling method.  The boxplots for the 
Whitewater and Zumbro Rivers looked similar to the Mississippi:  \\
*INSERT BOXPLOT FROM STATUS REPORT HERE  \\
We found no obvious trend indicating one site size to be more independent than 
the others.  

	While this is an interesting observation, we would like to be able to 
quantify it better by finding the distribution of our statistic.  This would 
allow us to make statements about how independent the data is, and if the 
difference in independence between different types of data is significant.   
This statistic seems hard to get, but we considered a modification.  Let \(S\) be
the number of successive \(0\),\(1\)'s minus the number of successive \(1\),\(1\)'s
per flight.  Under the null hypothesis that each site is an independent Bernoulli
trial with probability \(p\) we were able to compute the mean and standard 
deviation of \(S\).  

	The mean ...

	The standard deviation...

	Simulations of \(S\) show that the distribution of \(S\) is roughly normal.  
So using our calculations for the mean and the standard deviation, we computed
a Z-score for each flight, where we use a plug-in estimate for \(p\) to be the 
proportion of \(1\)'s per flight.  We said that Z-scores with an absolute value 
greater than \(2\) suggest that the assumption of independence between sites is 
violated for that flight.  We found that, in general, alternating sites are more
"independent" than all sites as measured by this statistic.  However, alternating
sites do not guarentee independence.  For instance, six of eleven flights along 
the Zumbro River during 2003 still have significant Z-scores with alternating
sites.  There were no noticable trends for this statistic with respect to 
observer, days since snow, or snow event.

*FIGURES OR TABLES??? WHICH ONES?

	We did find interesting trends with regard to sampling method.  We think that 
the lack of independence results from strings of ones where an otter left a track
across multiple sites.  When alternating sites are used, the strings are cut in 
half and the occurrence of \(1\),\(1\)'s is lower while the occurence of \(0\),
\(1\)'s stays about the same.  It is noteworthy that virtually all the violations
of independence are in the direction of more \(1\),\(1\)'s than would be expected.
This reinforces our thoughts about successive \(1\)'s representing the same otter.
We will discuss later whether this violation of independence actually matters in 
our model, and how this might affect the estimation of actual occupancy rates.
 

   		Using s, where s = (\# of 0-1's) - (\# of 1-1's) for a given flight.\\
          * Decided to use the z-score of s as a measure of independence.  \\
          * Therefore, needed the expected value and the standard deviation of s. \\
          * So, we needed the expected value of each term and the variance, which 
			required finding the variance of each term and the covariance between 
			the two terms. \\
          * Expected value is straightforward and we have all of the equations. \\
          * Variance is fun and we also have all of the relevant equations. \\
          * Calculated z-score for every flight and box-plotted them for each year
			and plot size and pitted data with every other plot against data with
			all plots. \\
          * Nothing important with respect to plot size. Alternating plots show 
			more independence than all plots. Look at report for more. \\


\section{Theoretical Stuff} - CHRISNA??
	\subsection{Bayesian Statistics}
	   	Bayes' Theorem \\
    	Useful for hierarchical modeling: look at section 3.3 in book. \\
    	Can use Markov Chain Monte Carlo (MCMC) to calculate posterior 
			distribution- bimodal issues.
	\subsection{BRugs/R}
    	All of our statistical analysis was performed in R.  \\
    	All of our modeling was performed using BRugs, which allowed us to use 
			WinBUGS commands and stuff in R.


\end{document}