\documentclass{article}
\usepackage{amsmath}
\usepackage{multirow}
\usepackage[dvipsnames, table]{xcolor}

\begin{document}

%%% *STANDARD HIERARCHICAL MODEL*:
\section{Standard Hierarchical Model}

    %% MODEL PARAMETERS:
    \subsection{Model Parameters}

    %% SIMULATED DATA, PT. 1:
    \subsection{Simulated Data}

    \noindent{\begin{tabular}{|l|l|l|l|}
        \hline
        \multicolumn{4}{|l|}{\textbf{The Model vs. Simulated Data}} \\
        \hline
            \(\psi\) & \(\theta\) & Convergence & 95\% CI (\(\psi\)) \\
        \hline
        \multirow{5}{*}{0.20}
            & \cellcolor{Green}0.20 & \cellcolor{Green}Good &
              \cellcolor{Green}(0.112, 0.305) \\
            & \cellcolor{Green}0.40 & \cellcolor{Green}Good &
              \cellcolor{Green}(0.142, 0.341) \\
            & \cellcolor{Green}0.60 & \cellcolor{Green}Good &
              \cellcolor{Green}(0.098, 0.284) \\
            & \cellcolor{Green}0.80 & \cellcolor{Green}Good &
              \cellcolor{Green}(0.097, 0.281) \\
            & \cellcolor{Yellow}1.00 & \cellcolor{Yellow}Okay &
              \cellcolor{Yellow}(0.117, 0.302) \\
        \hline
        \multirow{5}{*}{0.40}
            & \cellcolor{Yellow}0.20 & \cellcolor{Yellow}Okay &
              \cellcolor{Yellow}(0.388, 0.721) \\
            & \cellcolor{Green}0.40 & \cellcolor{Green}Good &
              \cellcolor{Green}(0.270, 0.494) \\
            & \cellcolor{Green}0.60 & \cellcolor{Green}Good &
              \cellcolor{Green}(0.287, 0.512) \\
            & \cellcolor{Red}0.80 & \cellcolor{Red}Bad &
              \cellcolor{Red}(0.277, 0.507) \\
            & \cellcolor{Yellow}1.00 & \cellcolor{Yellow}Okay &
              \cellcolor{Yellow}(0.279, 0.503) \\
        \hline
        \multirow{5}{*}{0.60}
            & \cellcolor{Yellow}0.20 & \cellcolor{Yellow}Good &
              \cellcolor{Yellow}(0.312, 0.581) \\
            & \cellcolor{Green}0.40 & \cellcolor{Green}Good &
              \cellcolor{Green}(0.400, 0.630) \\
            & \cellcolor{Yellow}0.60 & \cellcolor{Yellow}Okay &
              \cellcolor{Yellow}(0.462, 0.687) \\
            & \cellcolor{Yellow}0.80 & \cellcolor{Yellow}Okay &
              \cellcolor{Yellow}(0.502, 0.727) \\
            & \cellcolor{Red}1.00 & \cellcolor{Red}Bad &
              \cellcolor{Red}(0.318, 0.549) \\
        \hline
        \multirow{5}{*}{0.80}
            & \cellcolor{Green}0.20 & \cellcolor{Green}Good &
              \cellcolor{Green}(0.578, 0.838) \\
            & \cellcolor{Green}0.40 & \cellcolor{Green}Good &
              \cellcolor{Green}(0.736, 0.915) \\
            & \cellcolor{Yellow}0.60 & \cellcolor{Yellow}Okay &
              \cellcolor{Yellow}(0.587, 0.829) \\
            & \cellcolor{Yellow}0.80 & \cellcolor{Yellow}Okay &
              \cellcolor{Yellow}(0.599, 0.811) \\
            & \cellcolor{Red}1.00 & \cellcolor{Red}Bad &
              \cellcolor{Red}(0.612, 0.830) \\
        \hline
        \multirow{5}{*}{1.00}
            & \cellcolor{Yellow}0.20 & \cellcolor{Yellow}Good &
              \cellcolor{Yellow}(0.895, 0.999) \\
            & \cellcolor{Yellow}0.40 & \cellcolor{Yellow}Okay &
              \cellcolor{Yellow}(0.927, 0.999) \\
            & \cellcolor{Green}0.60 & \cellcolor{Green}Good &
              \cellcolor{Green}(0.949, 1.000) \\
            & \cellcolor{Yellow}0.80 & \cellcolor{Yellow}Good &
              \cellcolor{Yellow}(0.802, 0.954) \\
            & \cellcolor{Red}1.00 & \cellcolor{Red}Bad &
              \cellcolor{Red}(0.800, 0.955) \\
        \hline
    \end{tabular}}

%%% *DISCUSSION*:
\section{Discussion}

    %% GETTING THE ACTUAL DATA TO CONVERGE:
    \subsection{Obtaining Convergence on the Actual Data}
    Recall that our standard hierarchical model contains the following
    parameters: \(\psi\) (the rate of occupancy), \(\theta\) (the probability of
    laying down a track given occupancy), \(p\) (the probability of detecting a
    track given that it is present, which is unique to each observer and makes
    \(1-p\) the false negative rate of detecting a track), and \(E\) (the false
    positive rate of detecting a track, which is unique to each observer).

    Once both the standard hierarchical model and the addition of a CAR prior
    distribution on \(\theta\) to the standard model exhibited convergence on
    our simulated data, we moved onto running both models on the data given to
    us by the Minnesota DNR.  Recall that data was gathered during the winters
    of 2003 and 2004 along the Mississippi, Whitewater, and Zumbro Rivers and
    aggregated with ArcGIS into vectors containing ones and zeroes representing
    the detection and non-detection of a track, respectively, within plots of
    length 402 meters, 804 meters, and 1608 meters.  The covariates recorded
    were: the observer, the date, and the number of days since the last snow
    event.  In general, the sparseness of the data gathered in 2004 caused us to
    simplify the model in order to obtain results.  Unfortunately, the
    simplification results in estimates of \(\psi\) that do not seem accurate
    and are not directly comparable to estimates of \(\psi\) on 2003's data.

    The standard model converged reliably on all data gathered in 2003, but it
    did not converge on any of the data gathered in 2004.  Considering that the
    amount of data gathered in 2004 pales in comparison to the amount of data
    gathered in 2003, this result was not surprising.  The vast majority of
    2004's data consists of just one observer taking observations once after
    each snow event, which causes \(p\) and \(E\) in particular to not converge,
    regardless of plot size.

    To obtain convergence on 2004's relatively sparse data, we removed \(E\),
    which adds to the model the assumption that there are no false positives
    and, consequently, simplifies it greatly.  Although the model still did not
    converge on data with plot sizes of 804m and 1608m, both the Whitewater and
    Zumbro Rivers' data at the 402m plot size converged.  Although none of the
    Mississippi River's 2004 data converged, we still include estimates of
    \(p\), \(\psi\), and \(\theta\) at the 402m plot size to accompany the
    estimates corresponding to the Whitewater and Zumbro Rivers.  For the most
    part, however, we feel that the absence of \(E\) from the model makes the
    the estimates of \(\psi\) much less dependable.

    The addition of a CAR prior distribution on \(\theta\) to the standard model
    caused the model to fail to converge on all of the data.  Although the
    standard model and the CAR model performed similarly on simulated data
    whenever both models converged, the standard model converged much more often
    than did the CAR model.  This is most likely due to the fact that the CAR
    model attempts to estimate too many parameters.  Whereas the standard model
    estimates between eight and twelve parameters on the actual data (\(\psi\),
    \(\theta\), the four parameters of the beta distributions corresponding to
    \(p\) and \(E\), \(p\) for each observer, and \(E\) for each observer), the
    CAR model estimates \(\theta\) for each site.  Because there are at least
    thirty-five sites in each data set (regardless of plot size), the CAR prior
    distribution adds a considerable amount of complexity to the model.

    We also tried to simplify this model by eliminating \(E\) because this was
    somewhat successful with respect to convergence with the standard model.  We
    first attempted to fit this model to data from 2003 at the 402m plot size.
    Although the model successfully converged on all of this data, its estimates
    of \(\psi\), all of which were greater than 0.9, did not seem reasonable,
    especially in comparison to the standard model's estimates when given the
    same data.  As a result, we did not attempt to fit this model to the rest of
    the data and do not include this model's estimates within our results.

    %% RESULTS:
    \subsection{Results}
    Below, in tables, lie our estimates for all of the Minnesota DNR's data from
    2003 and 2004.  Below each table is detailed commentary about the ability of
    the standard hierarchical model to convergence to the corresponding data.

    With respect to the model's ability to converge on the actual data, the most
    important factors are plot size, the number of flights recorded, and the
    number of estimated parameters.  In general, both plot size and the number
    of flights recorded must be sufficiently small and large, respectively, for
    the standard model with \(E\) to converge.  

    2003's data worked relatively well with our model.  The presence of numerous
    observations that occur in the second and third days since the most recent
    snow event allows one of the more crucial assumptions of our hierarchical
    model -- that otter tracks laid on each day after a snow event are visible
    throughout the entire observation period and therefore, that it is more
    likely for an observer to detect a track on later days -- to be thoroughly
    tested.

    2004's data, which suffered from the lack of multiple observers and
    observations later than one day after each snow event, does not provide
    enough data with respect to time (measured by the number of days since the
    last snow event) for the parameters \(p\) and \(E\) to converge, regardless
    of the number of Markov Chain Monte Carlo (MCMC) iterations performed.

    % Mississippi River!
    \begin{center}
    \begin{tabular}{|l|l|l|l|}
        \hline
        \multicolumn{4}{|l|}{\textbf{Mississippi River, 2003}} \\
        \hline
            & 402m & 804m & 1608m \\
        \hline
        \multirow{2}{*}{E[Danny]}
            & 0.01954 & 0.04808 & 0.06885 \\
            & (0.006515, 0.03663) & (0.01874, 0.08187) & (0.01778, 0.1383) \\
        \hline
        \multirow{2}{*}{E[John]}
            & 0.05216 & 0.07771 & 0.13650 \\
            & (0.027710, 0.07745) & (0.03881, 0.12100) & (0.03449, 0.2330) \\
        \hline
        \multirow{2}{*}{E[Tom]}
            & 0.02668 & 0.06140 & 0.10230 \\
            & (0.011990, 0.04622) & (0.02789, 0.10390) & (0.04287, 0.1796) \\
        \hline
        \multirow{2}{*}{p[Danny]}
            & 0.65240 & 0.67340 & 0.74960 \\
            & (0.548300, 0.75990) & (0.56760, 0.78290) & (0.60510, 0.8910) \\
        \hline
        \multirow{2}{*}{p[John]}
            & 0.67770 & 0.77580 & 0.82540 \\
            & (0.583500, 0.77280) & (0.66060, 0.88690) & (0.71390, 0.9326) \\
        \hline
        \multirow{2}{*}{p[Tom]}
            & 0.53670 & 0.54220 & 0.65330 \\
            & (0.501200, 0.61300) & (0.50160, 0.62850) & (0.51530, 0.8104) \\
        \hline
        \multirow{2}{*}{\(\psi\)}
            & 0.59380 & 0.69370 & 0.75820 \\
            & (0.439000, 0.75990) & (0.51800, 0.87380) & (0.54110, 0.9300) \\
        \hline
        \multirow{2}{*}{\(\theta\)}
            & 0.15280 & 0.20730 & 0.27690 \\
            & (0.111900, 0.19780) & (0.14580, 0.27390) & (0.19030, 0.3716) \\
        \hline
    \end{tabular}
    \end{center}
    
    This data converged well at the 402m and 804m plot sizes.  At the 1608m plot
    size, \(E\), \(\psi\), and \(\theta\) did not converge well.

    \begin{center}
    \begin{tabular}{|l|l|}
        \hline
        \multicolumn{2}{|l|}{\textbf{Mississippi River, 2004}} \\
        \hline
            & 402m \\
        \hline
        \multirow{2}{*}{p[Danny]}
            & 0.3129 \\
            & (0.2174, 0.8107) \\
        \hline
        \multirow{2}{*}{\(\psi\)}
            & 0.7613 \\
            & (0.6609, 0.8833) \\
        \hline
        \multirow{2}{*}{\(\theta\)}
            & 0.8511 \\
            & (0.2426, 0.9973) \\
        \hline
    \end{tabular}
    \end{center}

    This data did not converge.  We do not think that these estimates are
    accurate, but they are included for the sake of completeness.
    
    % Whitewater River!
    \begin{center}
    \begin{tabular}{|l|l|l|}
        \hline
        \multicolumn{3}{|l|}{\textbf{Whitewater River, 2003}} \\
        \hline
            & 402m & 804m \\
        \hline
        \multirow{2}{*}{E[Danny]}
            & 0.05608 & 0.0555 \\
            & (0.02474, 0.09554) & (0.002859, 0.1390) \\
        \hline
        \multirow{2}{*}{E[John]}
            & 0.16570 & 0.1931 \\
            & (0.10130, 0.23510) & (0.065910, 0.3384) \\
        \hline
        \multirow{2}{*}{E[Tom]}
            & 0.15350 & 0.2241 \\
            & (0.09758, 0.21660) & (0.110000, 0.3623) \\
        \hline
        \multirow{2}{*}{p[Danny]}
            & 0.66990 & 0.6985 \\
            & (0.56080, 0.79040) & (0.573100, 0.8366) \\
        \hline
        \multirow{2}{*}{p[John]}
            & 0.92040 & 0.9877 \\
            & (0.84270, 0.97590) & (0.945100, 1.0000) \\
        \hline
        \multirow{2}{*}{p[Tom]}
            & 0.75910 & 0.8575 \\
            & (0.65060, 0.85800) & (0.745400, 0.9600) \\
        \hline
        \multirow{2}{*}{\(\psi\)}
            & 0.87590 & 0.9126 \\
            & (0.72910, 0.98880) & (0.746900, 0.9955) \\
        \hline
        \multirow{2}{*}{\(\theta\)}
            & 0.24100 & 0.3779 \\
            & (0.17890, 0.30900) & (0.286500, 0.4721) \\
        \hline
    \end{tabular}
    \end{center}

    While the 402m data converged well, \(E\), \(\psi\), and \(\theta\) did not
    converge well for the 804m data.

    \begin{center}
    \begin{tabular}{|l|l|l|l|}
        \hline
        \multicolumn{2}{|l|}{\textbf{Whitewater River, 2004}} \\
        \hline
            & 402m \\
        \hline
        \multirow{2}{*}{p[Danny]}
            & 0.8237 \\
            & (0.5323, 0.9804) \\
        \hline
        \multirow{2}{*}{\(\psi\)}
            & 0.6066 \\
            & (0.4246, 0.8382) \\
        \hline
        \multirow{2}{*}{\(\theta\)}
            & 0.2068 \\
            & (0.1211, 0.3550) \\
        \hline
    \end{tabular}
    \end{center}

    This data converged well.  However, we feel that the absence of \(E\) from
    the model makes the estimates much less dependable.  The estimates are also
    much less precise than those from 2003 -- the confidence intervals of
    \(\psi\) and of \(p[Danny]\) are twice as large as those from 2003.

    % Zumbro River!
    \begin{center}
    \begin{tabular}{|l|l|l|}
        \hline
        \multicolumn{3}{|l|}{\textbf{Zumbro River, 2003}} \\
        \hline
            & 402m & 804m \\
        \hline
        \multirow{2}{*}{E[Danny]}
            & 0.02922 & 0.007066 \\
            & (0.006786, 0.05994) & (0.0000, 0.03403) \\
        \hline
        \multirow{2}{*}{E[John]}
            & 0.03388 & 0.008312 \\
            & (0.008584, 0.06880) & (0.0000, 0.04188) \\
        \hline
        \multirow{2}{*}{E[Tom]}
            & 0.02228 & 0.012960 \\
            & (0.002430, 0.05385) & (0.0000, 0.05642) \\
        \hline
        \multirow{2}{*}{p[Danny]}
            & 0.81890 & 0.837100 \\
            & (0.749500, 0.88540) & (0.7615, 0.90700) \\
        \hline
        \multirow{2}{*}{p[John]}
            & 0.81770 & 0.852500 \\
            & (0.740100, 0.88670) & (0.7597, 0.92230) \\
        \hline
        \multirow{2}{*}{p[Tom]}
            & 0.73120 & 0.805400 \\
            & (0.625400, 0.82440) & (0.6885, 0.89780) \\
        \hline
        \multirow{2}{*}{\(\psi\)}
            & 0.66550 & 0.651600 \\
            & (0.549100, 0.78150) & (0.5098, 0.78210) \\
        \hline
        \multirow{2}{*}{\(\theta\)}
            & 0.29950 & 0.395800 \\
            & (0.236400, 0.36550) & (0.3100, 0.48520) \\
        \hline
    \end{tabular}
    \end{center}

    Again, the 402m data converged well.  With respect to the 804m data, while
    \(\psi\) and \(\theta\) converged well, none of the other parameters did.

    \begin{center}
    \begin{tabular}{|l|l|l|l|}
        \hline
        \multicolumn{2}{|l|}{\textbf{Zumbro River, 2004}} \\
        \hline
            & 402m \\
        \hline
        \multirow{2}{*}{p[Danny]}
            & 0.9166 \\
            & (0.7739, 0.9918) \\
        \hline
        \multirow{2}{*}{\(\psi\)}
            & 0.4773 \\
            & (0.3586, 0.6114) \\
        \hline
        \multirow{2}{*}{\(\theta\)}
            & 0.4232 \\
            & (0.3073, 0.5520) \\
        \hline
    \end{tabular}
    \end{center}

    This data converged well.  However, as is the case with all of the 2004
    data, we feel that the absence of \(E\) from the model makes the estimates
    much less dependable.

    %%
    \subsection{The Effect of Plot Size}
    %%
    \subsection{All Plots vs. Every Other Plot}
    %%
    \subsection{How Important Are Our Covariates?}
    %%
    \subsection{Best Practices for Measuring Otter Occupancy}

\end{document}